\clearpage
\begin{centering}
\textbf{SUMMARY}\\
\vspace{\baselineskip}
\end{centering}



%\pagenumbering{gobble}  %remove page number on summary page

The purpose of this work is to investigate methods by which the quantum states of trapped ions may be better controlled and measured for the applications of quantum computing, quantum simulation, and the generation of atomic frequency standards. We report on two primary projects: first, we control the relative coupling strengths of carrier and first-order motional sideband interactions of a trapped ion by placing it in a resonant optical standing wave. Our configuration uses the surface of a microfabricated chip trap as a mirror, avoiding the technical challenges of in-vacuum optical cavities. Displacing the ion along the standing wave, we show a periodic suppression of the carrier and sideband transitions, with the cycles for the two cases 180$^{\circ}$ out of phase with each other. This technique allows for the suppression of off-resonant carrier excitations when addressing the motional sidebands, and has applications in quantum computation and quantum simulation. Second, we investigate methods proposed in Ref. \cite{BorregaardSorensen} which may improve the performance of trapped ion (or atom) frequency standards when the interrogating local oscillator is of limited quality. These methods could be applied to future generationss of small-scale frequency standards, which are in demand for applications such as local GPS networks and distributed sensors. We propose a scheme for  implementing these methods in a $^{40}$Ca$^+$ based frequency standard, and outline the technical requirements for doing so. Finally, we present preliminary results characterizing single-ion coherence times and ion ``state destruction'' events in our system, and discuss how they will affect the realization of the proposed scheme. 