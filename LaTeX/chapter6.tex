\chapter{Conclusion and future outlook}


\section{On motional state control via standing wave beams}

In Ch. 4, we have shown how the coupling strengths of carrier and first order sideband transitions in a trapped ion may be selectively controlled by displacing the ion within a resonant standing wave field. Using the surface of our Gen IIc ion trap as a mirror, we produced a standing wave field at the ion position by reflecting a 729 nm laser off the trap surface, thus avoiding the challenges of in-vacuum optical cavities. As predicted, the carrier and sideband coupling strengths demonstrate a periodic dependence on the ion's position within the standing wave fringes, with the cycles of the two cases 180$^{\circ}$ out of phase with each other. Despite having an imperfect standing wave due to an 18$^{\circ}$ angle in the incident beam and a limited reflectivity from the aluminum trap surface, we were able to achieve equivalent carrier and sideband suppressions of 8.4 dB and 11 dB, respectively. We calculate that a 29 dB equivalent carrier suppression is possible in our configuration with improved beam alignment, whereas in a gold coated trap this figure could be increased to $>$ 40 dB. Recently, we have become aware of a planar ion trap fabricated directly on top of a highly reflective (HR) UV mirror for the specific purpose of achieving a configuration similar to ours \cite{doi:10.1063/1.4970542}. Although the authors do not provide a number, it is reasonable to speculate that such an optimized material could be even more reflective than the gold trap, and could achieve even greater coupling strength suppressions.

Based on our results, we believe that our standing wave configuration is a viable alternative to in-vacuum cavities, and can be advantageous for various experiments in quantum computation and simulation. There are several potential applications of which we are already aware: first, suppression of the carrier allows the driving laser's power to be freely increased by an equivalent amount, allowing for faster sideband interactions with no increased chance of an off-resonant carrier excitation. Such increased speeds could be particularly beneficial in ion trap quantum computations, as they can increase the fidelity of two-qubit gate operations. For the 8.4 dB effective carrier suppression we achieve, sideband interactions could be performed 2.5 times faster; at the 29 dB suppression limit of our aluminum trap, 28 times faster; at the $>\,$40 dB suppression limit of a gold coated trap, $>\,$100 times faster. Second, a 29 dB equivalent carrier suppression factor achievable in our current configuration would reach the regime in which simulating the expansion of the universe with trapped ions becomes experimentally feasible. Per Ref. \cite{Menicucci10.NJP.12.095019}, suppressing the carrier allows for the excitation of detectors due to the creation of cosmic photons. When the beam has more running wave character, the simulation is dominated by excitation of detectors with photon creation or destruction. Third, standing wave configurations such as the one described here have also been considered for quantum phase transition experiments \cite{PhysRevLett.118.073001}.

%Suppression of the carrier implies that the driving laser's power may be freely increased by an amount equivalent to the effective suppression, allowing for faster sideband interactions with no increased chance of an off-resonant carrier excitation. For the 8.4 dB suppression we have achieved, sideband interactions could be performed 2.5 times faster; at the 29 dB suppression limit of our aluminum Gen IIc trap, 28 times faster; at the $>$ 40 dB suppression limit of a gold coated trap, $>$ 100 times faster. Performing faster sideband interactions could, in particular, improve two-qubit gate fidelities in ion trap quantum computers. 




\section{On the implementation of multiple ensembles and adaptive measurements in atomic clocks}

In Ch. 5, we have discussed the merits of multiple ensembles and adaptive measurement techniques in atomic clocks, per a proposal by Borregaard and S\o rensen \cite{BorregaardSorensen}, and have put forth a scheme for implementing these methods in a $^{40}$Ca$^+$ based clock. If successful, these methods would allow for improved clock performance when the local oscillator (LO) is of limited quality, as will inevitably be the case in a small-scale device. Realizing these methods in a laboratory scale experiment would be the first step towards adapting them for small-scale frequency standards, which are in high demand for applications such as GPS and local network synchronization. As we have discussed, we believe that $^{40}$Ca$^+$ based clocks haave unique advantages which make them a good candidate for small-scale integrated devices.   

In preliminary work towards realizing this proposed clock scheme, we have measured a single-ion coherence time of $\tau_2^* \approx 3.4$ ms in our system. Our scheme includes several ion transport operations which must be performed adiabatically, and the update rate of our trap electrodes limits these steps to durations of several ms at the least. As such, this coherence time is prohibitively short for our needs. However, our results suggest that the coherence time is limited by the linewidth of the 729 nm laser at the ion, due to either noise/imperfections in the cavity stabilization feedback loop, or to phase noise modulation occurring in fiber optic cables. Both of these issues are surmountable. Our high finesse cavity should be capable of 10-100 Hz linewidths with appropriate feedback settings, and phase noise modulations in optical fibers can be substantially reduced using a proven technique \cite{Ma:94}. Eliminating these issues should provide coherence times of 50-100 ms. This would be our first priority in moving forward with this project. In addition, it would be worthwhile to consider upgrading the system electronics to support faster electrode update rates. Transport and ion separation operations with marginal heating ($<$ 2-3 quanta) have been demonstrated on timescales of 10-50 $\mu$s when using an update rate of 50 MHz \cite{PhysRevLett.109.080502}.

We have also made preliminary characterizations/measurements of ion state destruction events in our system by using numerical simulations to parameterize experiment data. We find that non-trivial chances of state destruction occur when the 397 nm detection beam is pulsed in a variety of trap locations that are beyond the immediate vicinity of an ion undergoing interrogation. Because our proposed clock scheme includes detection events which occur separately while other ensembles are being interrogated, reducing state destruction would also be a top priority in the future of this project. Improvements could be immediately made by reducing the focused waist of the 397 nm beam to 10-15 $\mu$m, a simple matter of optics, and increasing the isolation of the PMT from ambient room light in order to improve detection efficiencies. Once these are done, more thorough measurements of state destruction should be made in order to determine how best to minimize its effects in our clock scheme. Furthermore, although we would wish to take steps to minimize state destruction, it is entirely likely that it cannot be eliminated entirely. Further work on this project would include numerical simulations to quantify how state destruction would limit clock performance. Such efforts have already been started by other members of our group working in parallel on a yitterbium based clock. 

Based on these preliminary results, we remain confident that a 2-ensemble clock with adaptive measurements can be realized in our  $^{40}$Ca$^+$ system. With $N \geq 4 \ (7)$ ions per ensemble, Eq. \ref{eq:clockstab} predicts improved performance over a single ensemble of $2 \times N$ ions for white ($1/f$) LO noise. This prediction assumes that $\gamma T_{1,max} = 0.3$ as a result of the adaptive measurements. Even if the implementation is not perfect and $\gamma T_{1,max} < 0.3$, or if the measured performance is otherwise less than predicted due to unanticipated reasons, $N$ can be increased accordingly to compensate for performance shortcomings.  