\chapter{Introduction and Background}

Within the last several decades it has become possible to control and measure the state of various quantum systems with an increasing level of sophistication \cite{doi:10.1080/09500349708231863, QuantMeas}. In addition to providing the experimental basis to test fundamental aspects of quantum theory \cite{QuantTh}, practical applications of these techniques have given rise to the concept of ``quantum technology'' for uses such as sensing, metrology, microscopy, computation, simulation, information processing, cryptography, electronics and photonics, frequency standard generation, etc \cite{QuantMeas, Dowling1655}. Trapped ion systems are one choice of system that can realize many of these applications. Although they are well researched and offer a good degree of state control and measurement, there remain a variety of technical challenges and limitations which must be accounted for when using trapped ions as a platform for such technologies. The purpose of this work is to investigate methods by which the quantum state of trapped ions may be better controlled and measured for the particular applications of quantum computing, quantum simulation, and the generation of atomic frequency standards.

\section{On motional state control for computation and simulation}

The design of electromagnetic ion traps such as Penning or Paul traps leads to confining potentials that are effectively harmonic. As we'll see in the technical review provided in Ch. 2, the internal electronic states of trapped ions become coupled with the motional states when the ions interact with resonant light fields. This allows the motional states to be coherently controlled, leading to a variety of important applications. For instance, the coupling takes the form of an almost ideal Jaynes-Cummings interaction \cite{Jaynes63.IEEE.51.89}, allowing trapped ion systems to function as a platform for a variety of cavity QED experiments \cite{CavityQED,Wu97.PRL.78.3086}, as well as enabling the analysis of various non-classical motional states (such as Fock and squeezed states) \cite{PhysRevA.49.1202,PhysRevLett.76.1796}. As another example, the ability to coherently control the motional states forms the basis for two-qubit gates in ion trap quantum computation \cite{Cirac95.PRL.74.4091,Sorensen00.PRA.62.022311,Schmidt03.Nature.422.408}. 

Control of the motional states is often achieved by interacting with the ion on its motional sidebands. These interactions are also used for cooling ions to the motional ground state \cite{Diedrich89.PRL.62.403}, measuring ion heating rates \cite{Turchette.PRA.61.063418,Shu14.PRA.89.062308}, and identifying and cooling molecular ions \cite{Goeders13.JPhysChemA.117.9725,Rugango15.NJP.17.035009,Wan15.PRA.91.043425}. The speed at which these interactions can take place depends on the intensity of the exciting radiation. Due to spectral broadening, off-resonant coupling to the primary transition (evident either as motion independent population transfer or an AC Stark shift), places a limit on the speed of the sideband interactions. Suppressing the primary transition, referred to also as the carrier, can remove this limit and, in particular, would allow for improved two-qubit gate fidelities as the gate time becomes comparable or shorter than a cycle of the harmonic motion \cite{Sorensen00.PRA.62.022311,Mizrahi14.APB.114.45}. Suppression of the carrier also has applications in quantum simulation.  For instance, trapped ions have been proposed as a system for modeling the expansion of the universe \cite{Menicucci10.NJP.12.095019}. The simulation requires off-resonant excitation of both the red and blue sidebands by a red-detuned exciting field, with no coupling to the carrier. Because the blue sideband is both weaker and further from resonance than the carrier transition, suppression of the carrier is important for such an experiment. 

Replacing running wave optical beams with standing wave beams provides a method to selectively suppress the carrier and reduce off-resonant excitations when addressing the motional sidebands \cite{Cirac92.PRA.46.2668,Zhang12.PRA.85.053420}. In such a configuration, the coupling strengths of the carrier and sidebands acquire a periodic dependence on the atom's spatial position within the standing wave fringes \cite{Cirac92.PRA.46.2668,James98.APB.66.181}, with the cycles for the two cases $180^\circ$ out of phase with each other. This periodic dependence of the coupling strengths has been demonstrated in cavity experiments with trapped ions \cite{Leibrandt09.PRL.103.103001,Guthohrlein01.Nature.414.49,Steiner13.PRL.110.043003,PhysRevLett.89.103001}. However, the use of cavities involves technical challenges such as the alignment of optics in vacuum and the tendency for optics and dielectric mirrors to become charged, complicating the integration of cavities with microfabricated ion traps \cite{Harlander10.NJP.9.093035,Clark14.PRAppl.1.024004}.

In Ch. 4 of this work we demonstrate the same position dependence in a standing wave field produced by a single mirror, which in this case is simply the surface of a planar ion trap. This configuration has the advantage of being simpler to implement than an optical cavity. In order to account for imperfect beam alignment, reflection losses, and similar system limitations, we extend the calculations of Refs.~\cite{Cirac92.PRA.46.2668,James98.APB.66.181} to the case of non-normal incidence laser beams and unequal couplings of the incident and reflected laser beams with the ion. In doing so, we find a criterion for the out of phase carrier and sideband coupling strengths that is set by the incident angle of the laser beam and the orientation of the ion's harmonic motion. Furthermore, by using the position dependence of the coupling strengths within the standing wave fringes, we are able measure the ion's relative position as a function of applied electric field in order to map the trapping potentials. These results are compared with those given by numerical models of the trap system.

\section{On the improvement of atomic frequency standards}

Atomic frequency standards (colloquially called atomic clocks) are the most accurate time and frequency standards known, and serve as references for GPS satellites, sensors, television broadcasts, and any other applications which require precision time keeping. These devices use radiation from a local oscillator (LO) to excite atomic transitions, allowing the LO to be stabilized to the transition resonance. The first such device to be built was completed in 1949 and was based on an ammonia absorption line at 24 GHz. Although it was successful as a proof-of-concept, its performance (measured by fractional frequency offsets $\Delta f / f_0$ of $10^{-7} - 10^{-8}$) could not match the best quartz standards at the time (with offsets around 2 \times 10^{-8}$) \cite{nistclocks, ammoniaclock}. By the mid 1950s, however, cesium beam standards based on the 9.2 GHz hyperfine transition were demonstrating fractional offsets of approximately $10^{-9}$ \cite{firstCs}, overtaking quartz oscillators. This marked the beginning of the atomic timekeeping era. Modern cesium standards used to define the SI second have fractional offsets on the order of $10^{-16}$ \cite{nistf2}, whereas frequency standards based on optical transitions (i.e. reference frequencies in the $10^{14}$ Hz range) have achieved stabilities of $\sigma_y (\tau) \sim 10^{-17} / \sqrt{\tau}$ \cite{OpticalClocks} ($\sigma_y$ being the Allan variance, which will be discussed in Ch. 5). Regardless of the choice in reference frequency,  devices on the bleeding edge of accuracy are limited to laboratory use due to their size and sensitivity to environmental factors such as vibration and temperature. These fixed-position standards are sufficient so long as the reference signal can be distributed beyond the lab. Currently this is done via global GPS signals, but the fragile nature of the GPS signal is a functional concern, particularly to the US armed forces. There is a recognized need to miniaturize these frequency standard technologies and make them sufficiently robust for field use in order to synchronize local GPS networks and distributed sensors. This demand for frequency references with reduced size, weight, and operating power has previously manifested in programs such as the DARPA IMPACT (BAA-08-32) and STOIC (BAA-14-41) programs. Most recently, the DARPA Atomic Clocks with Enhanced Stability (ACES) program (BAA-16-19) has sought an integrated clock in a package size of 50 cm$^3$ or less, with an operating power of 250 mW or less, and a stability $\sigma_y (\tau) < 10^{-11} / \sqrt{\tau}$ out to 10$^5$ s, matching the performance of a GPS locked reference or rack mount beam standard.

Trapped ion clocks are a good candidate for meeting such demands due to the demonstrated scalability of RF ion traps. For instance, in the QIS lab (where all of the work in this thesis was performed) we use planar chip traps on the order of 1 cm$^2$ that are designed in-house and fabricated at Georgia Tech using CMOS techniques (e.g. \cite{IonTrap}). However, the clock's performance depends both on the interrogation of the ions, which provides long term stability, and on the absence of noise in the LO, which provides the short term stability needed for interrogation. For references based on optical atomic transitions, the LO is a laser and is typically stabilized to an optical cavity. These cavities are generally heavy blocks of glass which provide rigidity against residual vibrational stress, and are otherwise isolated from environmental vibrations. Small package devices will require smaller, less stable cavities, resulting in much noisier LOs and diminished performance of the frequency reference. Microwave transitions can alternatively be used as reference frequencies, but such transitions (around $10^{10}$ Hz) are on the order of $10^4$ times less sensitive to fractional frequency shifts than optical transitions (around 10^{14}$ Hz). Maximizing the interrogation time of the atomic ensemble becomes critical in microwave based references in order to achieve maximum sensitivity. As in the optical case, the maximum interrogation time will be set by the LO performance. 

A proposal by Borregaard and S\o{}rensen \cite{BorregaardSorensen} describes how the performance of a trapped ion (or atom) clock may be improved when otherwise limited by a noisy LO. Specifically, their methods involve using multiple atomic ensembles in combination with adaptive measurement techniques in the trapped ion/atom clock in order to increase the interrogation time $T$ of the LO. This in turn  improves clock performance. In Ch. 5 of this work we investigate applying these methods to a $^{40}$Ca$^+$ based clock where the interrogating LO is a 729 nm laser (the same techniques will also be applicable to microwave based clocks). We consider the merits and technical requirements of such a system, and present a scheme of how it might be realized. Finally, with the goal of implementing this scheme, we report preliminary experimental data with regards to single-ion coherence times and state destruction during clock interrogations, and discuss how these technical barriers may be addressed.
